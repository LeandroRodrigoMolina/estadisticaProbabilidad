\documentclass[]{article}

%opening
\title{Resumen sobre estadística y probabilidad.}
\author{Leandro Molina}

\begin{document}

\maketitle

\begin{abstract}
	\section*{que}
\end{abstract}
\pagebreak

\section{Capitulo 1.}
\subsection{¿Por qué se debe estudiar estadística?}
Hay 3 motivos para el estudio de la estadística estos son:
\begin{enumerate}
	\item La primera razon, consiste en que la informacion numerica prolifera por todas partes. si revisas diarios o revistas contienen mucha cantidad de informacion numerica.
	\item Una segunda razon, es que las tecnicas de la estadistica se emplean para tomar decisiones que afectan la vida diaria, es decir, que incluyen en su bienestar.
	\item Una tercera razon, el conocimiento de sus metodos facilita la compresion de la forma en que se toman las decisiones y proporciona un entendimiento mas claro de como le afectan. 
\end{enumerate}
Al encarar la necesidad de tomar decisiones en las que tenes que saber hacer un analisis de datos resultara de utilidad. Con el fin de tomar una decision informada, sera necesario llevar a cabo lo siguiente para poder tomar una decision informada:
\begin{enumerate}
	\item Determinar si existe informacion adecuada o si requiere informacion adicional.
	\item Reunir informacion adicional, si se necesita, de manera que no se obtengan resultados erroneos.
	\item Resumir los datos de manera util e informativa.
	\item Analizar la informacion disponible.
	\item Obtener conclusiones y hacer inferencias al mismo tiempo que se evalua el riesgo de tomar una decision incorrecta.
\end{enumerate}
En resumen hay por lo menos tres razones para estudiar estadistica: 1) los datos proliferan por todas partes; 2) las tecnicas estadisticas se emplean en la toma de decisiones que influyen en su vida; 3) sin que importe la carrera que elija, tomara decisiones profesionales que incluyan datos.

\subsection{¿Que se entiende por estadística?}
Posee dos significados: su aceptacion mas comun, la estadistica se refiere a informacion numerica. Una coleccion de informacion numerica recibe el nombre de \textbf{estadisticas}. La informacion estadistica se presenta en forma grafica, es util porque capta la atencion del lector e incluye una gran cantidad de informacion. 
\begin{center}
\textbf{Estadistica:} Ciencia que recoge, organiza, presenta, analiza e interpreta datos con el fin de propiciar una toma de decisiones mas eficaz.
\end{center}
\\El primer paso en el estudio de un problema consiste en recoger datos revelantes.
\end{document}
