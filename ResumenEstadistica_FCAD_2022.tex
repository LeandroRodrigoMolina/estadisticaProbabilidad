\documentclass[]{article}

%opening
\title{Resumen sobre estadística y probabilidad.}
\author{Leandro Molina}

\begin{document}

\maketitle

\begin{abstract}
	\section*{que}
\end{abstract}
\pagebreak

\section{Capitulo 1.}
\subsection{¿Que es la estadística?}
Hay 3 motivos para el estudio de la estadística estos son:
\begin{enumerate}
	\item La primera razon, consiste en que la informacion numerica prolifera por todas partes. si revisas diarios o revistas contienen mucha cantidad de informacion numerica.
	\item Una segunda razon, es que las tecnicas de la estadistica se emplean para tomar decisiones que afectan la vida diaria, es decir, que incluyen en su bienestar.
	\item Una tercera razon, el conocimiento de sus metodos facilita la compresion de la forma en que se toman las decisiones y proporciona un entendimiento mas claro de como le afectan. 
\end{enumerate}
Al encarar la necesidad de tomar decisiones en las que tenes que saber hacer un analisis de datos resultara de utilidad. Con el fin de tomar una decision informada, sera necesario llevar a cabo lo siguiente para poder tomar una decision informada:
\begin{enumerate}
	\item Determinar si existe informacion adecuada o si requiere informacion adicional.
	\item Reunir informacion adicional, si se necesita, de manera que no se obtengan resultados erroneos.
	\item Resumir los datos de manera util e informativa.
	\item Analizar la informacion disponible.
	\item Obtener conclusiones y hacer inferencias al mismo tiempo que se evalua el riesgo de tomar una decision incorrecta.
\end{enumerate}
\end{document}
