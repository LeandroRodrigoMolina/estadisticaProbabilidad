\documentclass[]{article}
\usepackage{graphicx}
\graphicspath{ {./images/} }
\usepackage[spanish]{babel}
%opening
\title{Resumen sobre estadística y probabilidad.}
\author{Leandro Molina}

\begin{document}

\maketitle

\begin{abstract}
	\section*{Resumen}
\end{abstract}
\pagebreak

\tableofcontents

\pagebreak
\section{Capitulo 1.}
\subsection{¿Por qué se debe estudiar estadística?}
Hay 3 motivos para el estudio de la estadística estos son:
\begin{enumerate}
	\item La primera razon, consiste en que la informacion numerica prolifera por todas partes. si revisas diarios o revistas contienen mucha cantidad de informacion numerica.
	\item Una segunda razon, es que las tecnicas de la estadistica se emplean para tomar decisiones que afectan la vida diaria, es decir, que incluyen en su bienestar.
	\item Una tercera razon, el conocimiento de sus metodos facilita la compresion de la forma en que se toman las decisiones y proporciona un entendimiento mas claro de como le afectan. 
\end{enumerate}
Al encarar la necesidad de tomar decisiones en las que tenes que saber hacer un analisis de datos resultara de utilidad. Con el fin de tomar una decision informada, sera necesario llevar a cabo lo siguiente para poder tomar una decision informada:
\begin{enumerate}
	\item Determinar si existe informacion adecuada o si requiere informacion adicional.
	\item Reunir informacion adicional, si se necesita, de manera que no se obtengan resultados erroneos.
	\item Resumir los datos de manera util e informativa.
	\item Analizar la informacion disponible.
	\item Obtener conclusiones y hacer inferencias al mismo tiempo que se evalua el riesgo de tomar una decision incorrecta.
\end{enumerate}
En resumen hay por lo menos tres razones para estudiar estadistica: 1) los datos proliferan por todas partes; 2) las tecnicas estadisticas se emplean en la toma de decisiones que influyen en su vida; 3) sin que importe la carrera que elija, tomara decisiones profesionales que incluyan datos.

\subsection{¿Que se entiende por estadística?}
Posee dos significados: su aceptacion mas comun, la estadistica se refiere a informacion numerica. Una coleccion de informacion numerica recibe el nombre de \textbf{estadisticas}. La informacion estadistica se presenta en forma grafica, es util porque capta la atencion del lector e incluye una gran cantidad de informacion. 
\begin{flushleft}
\textbf{Estadistica:} Ciencia que recoge, organiza, presenta, analiza e interpreta datos con el fin de propiciar una toma de decisiones mas eficaz.
\end{flushleft}
El primer paso en el estudio de un problema consiste en recoger datos revelantes. Estos deben organizarse de alguna forma y, tal vez, representarse en una grafica.
\subsection{Tipos de estadística.}
El estudio de la estadística se divide en dos categorias: la estadística descriptiva y la estadística inferencial.
\subsubsection*{Estadistica descriptiva.}
Es la ciencia que "recoge, organiza, presenta, analiza...datos". Esta parte de la estadistica recibe el nombre de \textbf{estadistica descriptiva}.

\begin{flushleft}
\textbf{Estadistica descriptiva:} Metodos para organizar, resumir y presentar datos de manera informativa.
\end{flushleft}
Se trata de estadistica descriptiva si calcula el crecimiento porcentual de una decada a otra. Sin embargo, no seria de naturaleza descriptiva si utiliza estos para el calcular con esos datos algo futuro.
Una masa de datos desorganizados resulta de poca utilidad. Las tecnicas de la estadistica descriptiva permiten organizar esta clase de atos y darles significado. Los datos se ordenan en una \textbf{distribucion de frecuencia} (mas adelante lo veremos). Se emplean diversas clases de \textbf{graficas} para describir datos.

\subsubsection*{Estadistica inferencial.}
La estadistica inferencial, tambien denominada \textbf{inferencia estadistica}. El principal interes que despierta esta disciplina se relaciona con encontrar algo relacionado con una poblacion a partir de una muestra de ella. Ya que estas son inferencias relacionadas con una poblacion, basadas en datos de la muestra, se trata de estadistica inferencial. Se podria considerar a la estadistica inferencial como la mejor conjetura que es posible obtener del valor de una poblacion sobre la base de la informacion de una muestra.
\begin{flushleft}
	\textbf{Estadistica inferencial: }Metodos que se emplean para determinar una propiedad de una \textbf{poblacion} con base en la informacion de una \textbf{muestra} de ella.
\end{flushleft}
Atencion a las palabras poblacion y muestra en la definicion de estadistica inferencial. Una \textbf{poblacion} puede constar de individuos, tambien puede consistir en objetos. Desde una perspectiva estadistica, una poblacion no siempre que tiene que ver con personas.

\begin{flushleft}
	\textbf{Poblacion: }Conjunto de individuos u objetos de interes o medidas que se obtienen a partir de todos los medios u objetos de interes.
\end{flushleft}
Con el objeto de inferir algo sobre una poblacion, lo comun es que se tome una muestra de ella.
\begin{flushleft}
	\textbf{Muestra: }Porcion o parte de la poblacion de interes.
\end{flushleft}
La toma de muestras para aprender algo sobre una poblacion es de uso frecuente en administracion, agricultura, politica y acciones de gobierno.

\subsection{Tipos de variables.}
Dos tipos basicos de variables: 1)Cualitativas y 2)Cuantitativas, la caracteristica que se estudia es de naturaleza no numerica, recibe el nombre de \textbf{variable cualitativa} o \textbf{atributo}. Cuando los datos son de naturaleza cualitativa, importa la cantidad o proporcion que caen dentro de cada categoria. Los datos cualitativos se resumen en tablas o graficas de barras. Cuando la variable que se estudia aparece en forma numerica, se le denomina \textbf{variable cuantitativa}. Las variables cuantitativas pueden ser discretas o continuas. Las \textbf{variables discretas} adoptan solo ciertos valores y existen vacios entre ellos. Las variables discretas son el resultados de una relacion numerica, las observaciones de una \textbf{variable continua} toman cualquier valor dentro de un intervalo especifico. Por lo general las variables continuas son el resultado de mediciones. \linebreak Resumen de los tipos de variables: \\
\includegraphics[width=14cm, height=8cm]{resumenTiposVariables1_2}

\subsection{Niveles de medición.}
Los datos se clasifican por niveles de medicion. El nivel de medicion de los datos rige los calculos que se llevan a cabo con el fin de resumir y presentar los datos. Tambien determina las pruebas estadisticas que se deben realizar. De hecho, existen cuatro niveles de medicion: nominal, ordinal, de intervalo y de razon. La medicion mas baja, o mas primaria, corresponde al nivel nominal. La mas alta, o el nivel que proporciona la mayor informacion relacionada con la observacion, es la medicion de razon.

\subsubsection*{Datos de nivel nominal.}
Las observaciones acerca de una variable cualitativa solo se clasifican y se cuentan. No existe una forma particular para ordenar las etiquetas, no existe un orden natural. Para el nivel nominal, la medicion consiste en contar, a veces, para una mejor compresión de lectura, estos conteos se convierten en porcentajes. Es necesario hacer que el porcentaje sume un total de 100\%, no existe un orden natural para los resultados. Para procesar datos a menudo se codifica la informacion en forma numerica. El nivel nominal tiene las siguientes propiedades:
\begin{enumerate}
	\item La variable de interes se divide en categorias o resultados.
	\item No existe un orden natural de los resultados.
\end{enumerate}

\subsubsection*{Datos de nivel ordinal.}
El nivel inmediato superior de datos es el \textbf{nivel ordinal}. No es posible distinguir la magnitud de las diferencias entre los grupos, ¿la diferencia entre superior y bueno es la misma que entre lo malo e inferior? No es posible afirmarlo. Las propiedades del nivel ordinal de los datos son las siguientes:
\begin{enumerate}
	\item Las clasificaciones de los datos se encuentran representadas por conjuntos de etiquetas o nombre (alto, medio, bajo), las cuales tienen valores relativos.
	\item En consecuencia, los valores relativos de los datos se pueden clasificar u ordenar.
\end{enumerate}

\subsubsection*{Datos de nivel de intervalo.}
El \textbf{el nivel de intervalo} de medicion es el nivel inmediato superior. Incluye todas las caracteristicas de nivel ordinar, pero, ademas, la diferencia entre valores constituye una magnitud constante. Si las distancias entre los numeros tienen sentido, aunque las razones no, entonces tiene una escala de intervalo de medicion. Las propiedades de los datos de nivel intervalo son las siguientes:
\begin{enumerate}
	\item Las clasificaciones de datos se ordenan de acuerdo con el grado que posea de las caracteristica en cuestion.
	\item Diferencias iguales en la caracteristica representan diferencias iguales en las mediciones.
\end{enumerate}

\subsubsection*{Datos de nivel de razón.}
Todos los datos cuantitativos son registrados en el nivel de razon de la medicion, el \textbf{nivel de razon} es el \textit{mas alto}. Posee todas las caracteristicas del nivel de intervalo, aunque, ademas, el punto 0 tiene sentido y la razon entre dos numeros significativa, si se encuentra en 0 significa la ausencia de algo (peso, dinero, etc). Las propiedades de los datos de nivel intervalo son las siguientes:
\begin{enumerate}
	\item Las clasificaciones de datos se ordenan de acuerdo con la cantidad de caracteristicas que poseen.
	\item Diferencias iguales en la caracteristica representan diferencias iguales en los numeros asignados a las clasificaciones.
	\item El punto cero representa la ausencia de caracteristicas y la razon entre dos numeros es significativa.
\end{enumerate}
La siguiente grafica resume las principales caracteristicas de los diversos niveles de medicion.
\\
\includegraphics[width=16cm, height=8cm]{resumenCaracteristicasNivelesMedicion1_3}
\section{Descripción de datos.}
{\large Tabla de frecuencias, distribuciones de frecuencias y su representación grafica.}
\subsection{Construcción de una tabla de frecuencias.}
La estadistica descriptiva se encarga de organizar datos con el fin de mostrar la distribucion general de estos y el lugar en donde tienden concentrarse, ademas de señalar valores de datos pocos usuales o extremos. El primer procedimiento que se emplea para organizar y resumir un conjunto de datos es una \textbf{tabla de frecuencias.}
\begin{center}
	\textbf{Tabla de frecuencias:} Agrupacion de datos cualitativos en clases mutuamente excluyentes que muestra el numero de observaciones en cada clase.
\end{center}
Recordar que, una variable cualitativa es de naturaleza no numerica; es decir, que la informacion es clasificable en distintas categorias. No hay un order particular en estas categorias. Por otro lado, estan las variables cuantitativas son de indole numerica.
\subsubsection*{Frecuencias relativas de clase.}
Es posible convertir las frecuencias de clase en frecuencias relativas de clase para mostrar la fraccion del numero total de observaciones en cada una de ellas. Una frecuencia relativa capta la relacion entre la totalidad de elementos de una clase y el numero total de observaciones. Para convertir una distribucion de frecuencias en una distribucion de frecuencias relativa, cada una de las frecuencias de clase se divide entre el total de observaciones.

\subsubsection*{Representación grafica de datos cualitativos.}
El instrumento mas comun para representar una variable cualitativa en forma grafica es la \textbf{grafica de barras}. En la mayoria de los casos, el eje horizontal muestra la variable de interes y el eje vertical la frecuencia o fraccion de cada uno de los posibles resultados. Una caracteristica distinta de esta herramienta es que existe una distancia o espacio entre las barras. Una grafica de barras es una representacion grafica de una tabla de frecuencias mediante una serie de rectangulos de anchura uniforme, cuya altura corresponde a la frecuencia de clase.
\begin{center}
	\textbf{Grafica de barras:} En ella, las clases se representan en el eje horizontal y la frecuencia de clase en el eje vertical. Las frecuencias de clase son proporcionales a las alturas de las barras.
\end{center}
Se muestra ejemplo de una grafica de barras en la siguiente pagina:\\
\includegraphics[width=13cm]{graficaBarras2_1.png} \\
Otro tipo de grafica util para describir informacion cualitativa es la \textbf{grafica de pastel}.
\begin{center}
	\textbf{Grafica de pastel: }Grafica que muestra la parte o porcentaje que representa cada clase del total de numeros de frecuencia.
\end{center}
El primer paso para elaborar una grafica de pastel consiste en registrar los porcentajes 0, 5, 10, 15, etc, de manera uniforme alrededor de la circunferencia de un circulo. El area rebanada representa alguna clase, cada rebanada de pastel representa la  porcion relativa de cada componente, es posible compararlas con facilidad.\\
Aca un ejemplo de grafica bizcochuelo.\\
\includegraphics{tablaFrecuenciasRelativas2_2.PNG}\\
\includegraphics[width=12cm, height=10cm]{graficoPastel2_2.png} \\
Las graficas de pastel y las de barras cumplen casi la misma funcion. ¿Cuales son los criterios para elegir una u otra? En la mayoria de los casos, las graficas de pastel son las mas informativas cuando se trata de comparar la diferencia relativa en el porcentaje de observaciones de cada uno de las variables de la escala nominal. Es preferible usar una grafica de barras cuando el objetivo es comparar el numero de observaciones en cada categoria.
\subsection{Construcción de distribuciones de frecuencias: datos cuantitativos.}

\end{document}
