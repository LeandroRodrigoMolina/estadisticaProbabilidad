\documentclass[]{article}

%opening
\title{Resumen sobre estadística y probabilidad.}
\author{Leandro Molina}

\begin{document}

\maketitle

\begin{abstract}
	\section*{que}
\end{abstract}
\pagebreak

\section{Capitulo 1.}
\subsection{¿Por qué se debe estudiar estadística?}
Hay 3 motivos para el estudio de la estadística estos son:
\begin{enumerate}
	\item La primera razon, consiste en que la informacion numerica prolifera por todas partes. si revisas diarios o revistas contienen mucha cantidad de informacion numerica.
	\item Una segunda razon, es que las tecnicas de la estadistica se emplean para tomar decisiones que afectan la vida diaria, es decir, que incluyen en su bienestar.
	\item Una tercera razon, el conocimiento de sus metodos facilita la compresion de la forma en que se toman las decisiones y proporciona un entendimiento mas claro de como le afectan. 
\end{enumerate}
Al encarar la necesidad de tomar decisiones en las que tenes que saber hacer un analisis de datos resultara de utilidad. Con el fin de tomar una decision informada, sera necesario llevar a cabo lo siguiente para poder tomar una decision informada:
\begin{enumerate}
	\item Determinar si existe informacion adecuada o si requiere informacion adicional.
	\item Reunir informacion adicional, si se necesita, de manera que no se obtengan resultados erroneos.
	\item Resumir los datos de manera util e informativa.
	\item Analizar la informacion disponible.
	\item Obtener conclusiones y hacer inferencias al mismo tiempo que se evalua el riesgo de tomar una decision incorrecta.
\end{enumerate}
En resumen hay por lo menos tres razones para estudiar estadistica: 1) los datos proliferan por todas partes; 2) las tecnicas estadisticas se emplean en la toma de decisiones que influyen en su vida; 3) sin que importe la carrera que elija, tomara decisiones profesionales que incluyan datos.

\subsection{¿Que se entiende por estadística?}
Posee dos significados: su aceptacion mas comun, la estadistica se refiere a informacion numerica. Una coleccion de informacion numerica recibe el nombre de \textbf{estadisticas}. La informacion estadistica se presenta en forma grafica, es util porque capta la atencion del lector e incluye una gran cantidad de informacion. 
\begin{flushleft}
\textbf{Estadistica:} Ciencia que recoge, organiza, presenta, analiza e interpreta datos con el fin de propiciar una toma de decisiones mas eficaz.
\end{flushleft}
El primer paso en el estudio de un problema consiste en recoger datos revelantes. Estos deben organizarse de alguna forma y, tal vez, representarse en una grafica.
\subsection{Tipos de estadística.}
El estudio de la estadística se divide en dos categorias: la estadística descriptiva y la estadística inferencial.
\subsubsection*{Estadistica descriptiva.}
Es la ciencia que "recoge, organiza, presenta, analiza...datos". Esta parte de la estadistica recibe el nombre de \textbf{estadistica descriptiva}.

\begin{flushleft}
\textbf{Estadistica descriptiva:} Metodos para organizar, resumir y presentar datos de manera informativa.
\end{flushleft}
Se trata de estadistica descriptiva si calcula el crecimiento porcentual de una decada a otra. Sin embargo, no seria de naturaleza descriptiva si utiliza estos para el calcular con esos datos algo futuro.
Una masa de datos desorganizados resulta de poca utilidad. Las tecnicas de la estadistica descriptiva permiten organizar esta clase de atos y darles significado. Los datos se ordenan en una \textbf{distribucion de frecuencia} (mas adelante lo veremos). Se emplean diversas clases de \textbf{graficas} para describir datos.

\subsubsection*{Estadistica inferencial.}
La estadistica inferencial, tambien denominada \textbf{inferencia estadistica}. El principal interes que despierta esta disciplina se relaciona con encontrar algo relacionado con una poblacion a partir de una muestra de ella. Ya que estas son inferencias relacionadas con una poblacion, basadas en datos de la muestra, se trata de estadistica inferencial. Se podria considerar a la estadistica inferencial como la mejor conjetura que es posible obtener del valor de una poblacion sobre la base de la informacion de una muestra.
\begin{flushleft}
	\textbf{Estadistica inferencial: }Metodos que se emplean para determinar una propiedad de una \textbf{poblacion} con base en la informacion de una \textbf{muestra} de ella.
\end{flushleft}
Atencion a las palabras poblacion y muestra en la definicion de estadistica inferencial. Una \textbf{poblacion} puede constar de individuos, tambien puede consistir en objetos. Desde una perspectiva estadistica, una poblacion no siempre que tiene que ver con personas.

\begin{flushleft}
	\textbf{Poblacion: }Conjunto de individuos u objetos de interes o medidas que se obtienen a partir de todos los medios u objetos de interes.
\end{flushleft}
Con el objeto de inferir algo sobre una poblacion, lo comun es que se tome una muestra de ella.
\begin{flushleft}
	\textbf{Muestra: }Porcion o parte de la poblacion de interes.
\end{flushleft}
La toma de muestras para aprender algo sobre una poblacion es de uso frecuente en administracion, agricultura, politica y acciones de gobierno.

\subsection{Tipos de variables.}
\end{document}
